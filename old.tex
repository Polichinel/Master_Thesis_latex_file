 \citep{King_Zeng_2001, king_zeng_2001b, Ward_Greenhill_Bakke_2010, Goldstone_2010, Schrodt_2014} 


Computational conflict prediction and forecasting is a natural extension of more traditional, estimation-focused, quantitative research on internal conflict and civil war. Early efforts such as \cite{Collier_Hoeffler_1998,Fearon_Laitin_2003,Collier_Hoeffler_2004,Hegre_Sambanis_2006} aimed at estimating the effects of various variables on the stability of countries with little or no attention payed to the prediction power of the model. These studies where centered on X so to say; estimating the effect of X (a given set of independent variables/features) on y (a given dependent variable/target). Due to a general critique of the methodology \citep{King_Zeng_2001, king_zeng_2001b, Ward_Greenhill_Bakke_2010, Goldstone_2010, Schrodt_2014} and parallel with the rapid development in statistics, machine learning, computation power and data-availability, the focus broadened to also included more y-centered approaches \citep{Goldstone_2010, perry_2013, chadefaux_2014, mueller_2016}.\par

%Furthermore the level of analysis move from country to more disaggregated sub national unites \citep{Cederman_Gleditsch_2009, Weidmann_2009, Hegre_Oestby_Raleigh_2009, Cederman_Buhaug_Roed_2009, Cederman_Weidmann_Gleditsch_2011, Cederman_Gleditsch_Buhaug_2013}.\par

Some scholars enhanced the X-centered studies with out-of-sample predictions of y as a better alternative to more traditional significance testing \citep{Goldstone_2010}, while others adopted purely y-centered approach, where the prediction of y was at the prime center of the effort \citep{perry_2013, chadefaux_2014, mueller_2016}.\par

Many of these new, more y-focused approaches, still include the same features as the more traditional X-focused studies. Some argue to have a more or less causal connection to conflict, such as poverty, deprivation, ethnic discrimination, natural resources, the nature of the political regime and population size \footnote{As presented in e.g. \cite{Collier_Hoeffler_1998, Fearon_Laitin_2003, Collier_Hoeffler_2004, Fearon_2004, Ross_2004, Hegre_Sambanis_2006, Goldstone_2010, Cederman_Gleditsch_Buhaug_2013}}. Others are more akin to early symptoms, such as specific sentiments in news outlets \cite{perry_2013, chadefaux_2014} or heightened intra-elite conflict \citep{Goldstone_2010}. Furthermore some pertains directly to the pattern of conflict as a function of itself, such as spatial conflict diffusion \citep{Goldstone_2010} [MORE H] and conflict traps or other kinds of temporal residues [H].\par


In a preliminary paper I have analyzed the predictive potential of various features pertaining to the first and last category just outlined \citep{Maase}. The aim of said paper was to map fertile paths for future research; not least this thesis. While features pertaining to poverty, deprivation, population size, and country size did contribute with relevant prediction power, the most important features - by far - pertained directly to the spatial/temporal pattern of conflict. Specifically three features: Distance from  the geografic unit in question to the nearest conflict, All past fatalities in the geographic unit, Number of past conflict years in geographic unit - all lagged one year \citep[17-18]{Maase}. The conclusion, as such, was that the most fertile path for further researches would focus on extracting even more information from these dimensions by developing more theoretically and methodically coherent features pertaining to these dimension \citep[21-23]{Maase}.\par



As such \cite{Hegre_Sambanis_2006} finds that having civil war in a neighboring country the previous year increases the risk of conflict \citep[521-529]{Hegre_Sambanis_2006}. \cite{Goldstone_2010} finds that sharing border with four or more conflict ridden countries increases a given countries risk of civil war and democratic breakdown significantly \cite[195]{Goldstone_2010}. 


%\todo[inline]{The Science: meget af det her skal ned i lit review. Her skal du komme med motivationen hurtigt så med løsningen og dit RC.}
%[ ]skriv når du har teori og metode kørende for dig. Som et condensat af disse afsnit; bare de mest motiverende dele; problemerne]
% Computational conflict prediction and forecasting is a natural extension of more traditional, estimation-focused, quantitative research on internal conflict and civil war. Early efforts such as \cite{Collier_Hoeffler_1998,Fearon_Laitin_2003,Collier_Hoeffler_2004,Hegre_Sambanis_2006} aimed at estimating the effects of various variables on the stability of countries with little or no attention payed to the prediction power of the model. These studies where centered on X so to say; estimating the effect of X (a given set of independent variables/features) on y (a given dependent variable/target). Due to a general critique of the methodology \citep{King_Zeng_2001, king_zeng_2001b, Ward_Greenhill_Bakke_2010, Goldstone_2010, Schrodt_2014} and parallel with the rapid development in statistics, machine learning, computation power and data-availability, the focus broadened to also included more y-centered approaches \citep{Goldstone_2010, perry_2013, chadefaux_2014, mueller_2016}.\par

% %Furthermore the level of analysis move from country to more disaggregated sub national unites \citep{Cederman_Gleditsch_2009, Weidmann_2009, Hegre_Oestby_Raleigh_2009, Cederman_Buhaug_Roed_2009, Cederman_Weidmann_Gleditsch_2011, Cederman_Gleditsch_Buhaug_2013}.\par

% Some scholars enhanced the X-centered studies with out-of-sample predictions of y as a better alternative to more traditional significance testing \citep{Goldstone_2010}, while others adopted purely y-centered approach, where the prediction of y was at the prime center of the effort \citep{perry_2013, chadefaux_2014, mueller_2016}.\par

% Many of these new, more y-focused approaches, still include the same features as the more traditional X-focused studies. Some argue to have a more or less causal connection to conflict, such as poverty, deprivation, ethnic discrimination, natural resources, the nature of the political regime and population size \footnote{As presented in e.g. \cite{Collier_Hoeffler_1998, Fearon_Laitin_2003, Collier_Hoeffler_2004, Fearon_2004, Ross_2004, Hegre_Sambanis_2006, Goldstone_2010, Cederman_Gleditsch_Buhaug_2013}}. Others are more akin to early symptoms, such as specific sentiments in news outlets \cite{perry_2013, chadefaux_2014} or heightened intra-elite conflict \citep{Goldstone_2010}. Furthermore some pertains directly to the pattern of conflict as a function of itself, such as spatial conflict diffusion \citep{Goldstone_2010} [MORE H] and conflict traps or other kinds of temporal residues [H].\par

% In a preliminary paper I have analyzed the predictive potential of various features pertaining to the first and last category just outlined \citep{Maase}. The aim of said paper was to map fertile paths for future research; not least this thesis. While features pertaining to poverty, deprivation, population size, and country size did contribute with relevant prediction power, the most important features - by far - pertained directly to the spatial/temporal pattern of conflict. Specifically three features: Distance from  the geografic unit in question to the nearest conflict, All past fatalities in the geographic unit, Number of past conflict years in geographic unit - all lagged one year \citep[17-18]{Maase}. The conclusion, as such, was that the most fertile path for further researches would focus on extracting even more information from these dimensions by developing more theoretically and methodically coherent features pertaining to these dimension \citep[21-23]{Maase}.\par

% Encouragingly, the features pertaining directly to the spatial/temporal pattern did not only carry much more prediction power than many of the more structural features; data pertaining to these dimension are also more readily available and more current. As an example; in \cite{Maase} I used the PRIO grid database to obtain the structural features at a disaggregated sub-national level \citep{Tollefsen_2012}. This database holds a plethora of interesting features in regards to conflict prediction aggregated from a legion of different sources. The geographic unites are constituted by a grid of cells each measuring $0.5\times0.5$ decimal degrees covering all of earth excluding Greenland and Antarctica \citep[XX]{Tollefsen_2012}. The database is indeed impressive, yet the challenge of using this data - and indeed most data like it - for forecasting, is that it takes researchers a lot of time to collected, process and make accessible. Both for the researches at PRIO and the researches responsible at each original source. At the time of writing - 2019 - the most up-to-date features have the last entry at 2015, while some have last entry at 2010; a five to ten year lag if I want to predict the conflict zones of 2020. This is less than optimal to say the least. Even if one where to bypass the Prio Grid database and go directly to the different original sources of data, much would still be lagging, and it would be a huge burden to collect, clean and merge these various data sources.\par

% Conversely, for data relating to the spatial/temporal pattern of conflict we only need one data source; on pertaining to when and where conflict has taken place in the past. Conveniently, it is naturally also the data source from which we get our target feature/dependable variable - but more importantly it more often updated and thus the last entry is more current. A number of good data sources on conflict exits, for now I use the Upsala Conflict Data Program(UCDP) \citep{Sundberg_2013, Croicu_Sundberg_2017}. At the time of writing the last entry is from 2017, and the entry for 2018 is due sometime doing the first half of 2019. As such, I might not be able to use data from 2019 to predict events in 2020, but I would be able to use data from 2018 to predict events in 2020 and onwards; a manageable two year lag.\par

% As noted, a full predictive framework such as the early-warning-system sought by the UN would naturally included data from both and more sources, but the focus of this thesis is only the development of a component pertaining to the temporal/spatial pattern.\par

% The challenge of this approach, is that conventional modelling of the temporal/spacial pattern of conflict has been somewhat methodological and theoretically underdeveloped - my own past contribution included. Let me exemplify by referencing a deterioration index recently proposed by \cite{perry_2013}. Perry's idea is related to the phenomenon of a conflict trap and the notion of some inertia in conflict which might deteriorate over time. Parry's time deteriorating index would be conceptualized by including the number of fatalities in some geographic unit for each of the last ten years as features. The deterioration rate is then incorporated by down-weighing these fatalities by dividing with the number of years past since the corresponding events. Thus, a feature pertaining to fatalities two years ago will have, as its values, half of the fatalities observed that year \cite[14]{perry_2013}. The sentiment and creativity presented here is rather symptomatically for the literature at large, and while the heart is at the right place, it is an ad hoc and underdeveloped solution. There is no reason to cap the effort at ten years and there is no theoretical or practical reason to choose the suggested deterioration rate. Instead of dividing with years past it might be more appropriate to divide by half that; or the deterioration rate might have an although different functional form, like an exponential or linear decay function - yet, if we do note know the function we should estimate it, not try to guess it. Indeed some scholars have used such slightly more sophisticated methods; \cite{Collier_Hoeffler_2004} uses a linear decay function counting years since last conflict and \cite{Hegre_Sambanis_2006} uses a linear decay function counting years since last peace. Yet, while such a framework comes closer to actual estimation, the specification of these functions still requires ad hoc and arbitrary specifications \cite[501]{Gelman_2013}. others, including myself, have used even less developed measurements; as such \cite{Cederman_Gleditsch_Buhaug_2013} simply count the number of previous conflicts in their geographical unit of analysis. Surely we can do better: If we do not know the functions or their hyper parameters they should be estimate. Furthermore, estimation at least gives us some indication of how good or bad the function we propose actually is.\par
% %And how are we to translate this index if we change time unite to months or weeks? 

% The exact same problem is present in the context of spatial diffusion. \cite{Goldstone_2010} finds that sharing border with four or more conflict ridden countries increases a given countries risk of civil war and democratic breakdown significantly \cite[195]{Goldstone_2010}. Again this appears theoretically underdeveloped; surely there must be differences in regards to the size of the bordering countries, the severity of the conflict and where in the neighbouring countries the conflicts occur. Furthermore time and space most interact in some way or another. Meaning that persisting conflict could be more likely to spread or, as argued by \cite{bara_2017}, the termination of conflict in one place might to push weapons and combatants to a neighbouring regions \cite[2003-2006]{bara_2017}. My own past contribution is no better; I used distance to nearest conflict without take into account the magnitude of this conflict, or the number of other adjacent conflicts. And even if I had included 2. nearest conflict, 3. nearest conflict ect, and scaled the distance by the some factor related to the fatalities count, I would not know if these conflicts surrounded the observation of interest or where all place neatly beside it. The pattern of conflict might matter; the magnitude of conflict might matter; the amount of adjacent conflicts might matter. Again the point is clear: modeling features to capture these phenomenons must be an estimation effort in it self. This is where the present thesis contributes.\par 

% Creating features out of raw data is referred to as feature engineering. In convectional machine learning this means modeling the data to maximize its predictive power. In political science this often means modeling the data to capture the theoretically mechanism connecting X to y \citep{Blimes_2006, Cederman_Gleditsch_Buhaug_2013}. E.g. change a wealth measure from absolute to relative if one believes relative deprivation to be more important the greed or state capacity. In the project at hand the feature engineering process will be an estimation effort  in itself; estimating how conflict move through time and space. This information will then be used in yet an other estimation effort; estimating the probability of future conflicts. This 'babushka doll' of estimations will understandably occur a bit exotic to many social scientist, but as I shall return to, this is routinely done in other fields.\par

% In the following subsection i will elaborate on this babushka doll, and clarify further the scope and bounds of the thesis at hand.\par


% THE SCIENCE SECTION CLIPS:


% -------------------
% hoved pointer fra "challenges."
% **disaggregated**

% In the conflict literature, civil war have often been dichotomously denoted; either a country is marked by civil war or it is not. Yet in reality civil war rarely encompasses entire countries, but are instead confined to specific regions of countries \cite[487]{Cederman_Gleditsch_2009}. However, using countries as unit of analysis inflicts a number of issues in regards to the study of internal conflicts.\par 

% However, as long as we treat internal conflict as a phenomenon which is necessarily country-wide we are bound to miss impotent nuances and patterns. Some regional conflicts will be presented as country-wide civil wars while serious regional conflicts will be treated as non-conflicts.

% Indeed the denomination "internal conflict" itself should compel us to explore the phenomenon at a sub-country level. As formulated by \cite{Cederman_Gleditsch_2009} "If our theories are disaggregated, then our empirical analyses and research designs
% should reflect this" \citep[490]{Cederman_Gleditsch_2009}. Encouragingly developments in statistic methods, computation power and data availability makes such endeavours evermore manageable \citep[446]{ol2010afghanistan}.\par

% As such, utilizing a disaggregated approach allows me to analyze the local temporal and spatial dynamics of conflicts at level more appropriate given the theoretical foundation \citep[446]{ol2010afghanistan}. Furthermore this also allows me to generate forecasting pertaining to specific geographic regions rather then entire countries. A potential powerful policy tool indeed.\par 


% **Forecasting**

% With very few exception, none of these studies attempted to test whether or not their proposed model was capable of predicting future conflicts. The focus was estimating the exact effect of a given feature on conflict-risk while proving or disproving the statistical significance of said effect. This approach however, was consistently criticized \citep{King_Zeng_2001, king_zeng_2001b, Ward_Greenhill_Bakke_2010, Goldstone_2010, Schrodt_2014}. Two main points can be derived from the criticism. First, even if forecasting is not the goal of a given study, significance testing is a misleading and flawed way to evaluate the included features of interest; prediction-power is a better benchmark for the salience of the included features, even when forecasting is not the end-goal \citep{Ward_Greenhill_Bakke_2010, Schrodt_2014}. Second, creating models with actual prediction-power will provide heuristic tools and powerful policy-guides for real world application \citep[372]{Ward_Greenhill_Bakke_2010}.\par

% Adopting such evaluation frameworks, some scholars have started using prediction to evaluated the salience of given features along with traditional parameter estimation \citep{Goldstone_2010}. Others abandoned parameter estimation in favour of more modern machine learning techniques, still utilizing the traditional roster of structural features but focusing solely on creating a predictive tool for policy recommendation \citep{perry_2013}. And lastly some abandoned both parameter estimation and the traditional roster of features all together, such as \cite{mueller_2016} who realies solely on text data from news outlets to predict future conflicts.\par


% **Spatial**

% Above I described that the temporal and spatial patterns of conflicts appears to hold substantial prediction power pertaining to further conflict zones compared to structural features. An other very important property is that the data pertaining to the patterns of conflict is more readily available, more up-to-data and more current the structural data. To capture the spatial and temporal patterns of conflict I only need event data. That is, data pertaining the past conflicts themselves. As I shall elaborate on in \autoref{data}, structural data such as wealth measures often takes a lot of time to gather and process. The consequence is that the last observations of such data is often rather dated by the time the data is made public. Event data such as that used in this thesis is released much more frequently with the last entry being much more current. As such there is amble reason to utilize this resource in the creation of a early warning system.


% Previous research on conflict diffusion has established that internal conflicts often unfolds near boarders \cite[29-30]{Blattman_Miguel_2010} and are often characterized by diffusing seamlessly across any administrative boundaries their might encounter \cite[442-443]{ol2010afghanistan}. Furthermore, it has also been firmly established that conflicts cluster in time and space, with \cite{crost2015conflict} listing no less then 20 published academic papers supporting this assertion \citep[15]{crost2015conflict}. Lastly, the emerging consensus is that the diffusion of conflict is often a direct product of contagion rather then merely a bi-product generated through clusters of structural features such as poverty or political regimes \citep{buhaug2008contagion,schutte2011diffusion,crost2015conflict,bara_2017}\footnote{though see \cite{black2013}.}. As such, efforts to take the spatial and temporal dimensions of conflict into account have gain increasing attention in estimation and prediction efforts alike. The challenge, however, is that conventional modelling of the temporal/spacial pattern of conflict often appears ad hoc and both methodological and theoretically underdeveloped - my own past contribution included.\par


% The point, naturally, is that if we do not know the relevant functions, effort should be made towards estimating them rather then guessing.\par 

% Naturally, moving from country level to a more disaggregated level comes with it own challenges. As such, the "bad neighborhood dichotomy" is appears even less appropriate. At a highly disaggregated level 2nd, 3th, 4th and potential n'th order neighbors would have to be considered preferable with some demising influence as a function of distance. \cite{ol2010afghanistan} does a commendable job of showing the diffusion of conflict from Afghanistan to Pakistan over the Duran Line from a disaggregated perspective. Yet they only choose to include 1st-order neighbors \cite[146]{ol2010afghanistan}. My own past contribution is hardly any better; I used distance to nearest conflict without take into account the magnitude of this conflict, or the number of other adjacent conflicts. And even if I had included 2. nearest conflict, 3. nearest conflict ect, and scaled the distance by the some factor related to the fatalities count, I would still not know if these conflicts surrounded and engulfed the observation of interest - or instead clustered neatly and distinctly beside it. The pattern of conflict might matter; the magnitude of conflict might matter; the magnitude of adjacent conflicts in 1st, 2nd and nth order might matter. Again the point is clear: modeling features to capture these phenomena must be an estimation effort in it self. Utilizing the insight presented in this, and the two preceding subsections, the next section will present a framework capable of exactly such endeavour.\par
